\chapter{Conclusions and Future Work}\label{ch:conclusions}

In dieser Arbeit haben wir uns mit vier konkreten Herausforderungen befasst, um ein GLOSA-System für Radfahrer umzusetzen: traffic light prediction, traffic light matching, accurate bike routing, und speed advisory.

Die erste Herausforderung war, an Ampeldaten zu gelangen und das Schaltverhalten von Ampeln zuverlässig zu prognostizieren. Bei der Akquise von Ampeldaten sind wir auf eine mitunter sehr heterogene Landschaft von aktuell verfügbaren Systemen und gerade erforschten Konzepten gestoßen. Während die Forschung an dezentralen Ampeldatensystemen derzeit mangels Smartphone-Kompatibilität weitestgehend an Fahrradfahrern vorbei geht, bleiben nur zentrale Dateninfrastrukturen als Variante übrig. Eine solche zentralisiert Ampeldaten sammelnde und zur Verfügung stellende Plattform, die Traffic Lights Data Plattform von Hamburg, wurde in dieser Arbeit verwendet. Unsere Langzeitstudie zeigt, dass es jedoch mitunter deutliche Herausforderungen mit der Qualitätssicherung einer solchen Datenplattform geben kann. Konkrete Untersuchungen der großflächigen Datenstabilität sowie ein entwickeltes Monitoring konnten verwendet werden, um systematische Probleme in der Dateninfrastruktur zu identifizieren und anschließend zu lösen. Speziell einzelne Verkehrsrechner, über welche Daten ausgeleitet wurden, schienen als problematisches Glied in der Datenkette immer wieder für Ausfälle zu sorgen.

Eine absolut verlässliche und stets verfügbare Prognose ist technisch nicht nur aus diesen Gründen schwer machbar. Hinzu kommen auch verkehrsadaptive Ampeln, welche durch ihr spontaneres Umschalten die Prognose beeinträchtigen können. In dieser Arbeit wurde statt einem aktuelleren selbstadaptiven Prognoseverfahren, beispielsweise auf Basis von Machine Learning, ein probabilistisches Prognoseverfahren aus einer anderen Arbeit wiederverwendet, welches bei verkehrsabhängigen Schaltungen zwischen sicheren und unsicheren Bereichen der Prognose unterscheiden kann. Dieses Verfahren konnte in verwandten Arbeiten bereits zeigen, dass es für instabile und zum Teil stark verzögerte Daten eine gute Option darstellt, und zusätzlich auch die Unsicherheitsbereiche der Prognose ohne starke zeitliche Fluktuation abbildet. Die Integration des Verfahrens zeigt, bis auf die bereits erwähnten Datenausfälle, dass das Verfahren an den meisten Ampeln eine gute Prognosequalität erreichte. 

Diese Erkenntnis war überraschend, da in verganenen Arbeiten häufig motiviert wurde, dass eine hohe Anzahl verkehrsabhängiger Ampeln auch zugleich eine schlechte Prognostizierbarkeit mit sich bringt. Speziell in Hamburg, in dem die meisten Ampeln zumindest die Fähigkeit besitzen, verkehrsadaptiv zu schalten, hätten wir daher eine schlechtere Performance erwartet. Um diese Erkenntnis weiter zu verfolgen und besser zu verstehen, welche verkehrsabhängigen Schaltmuster sich tatsächlich in den Ampeldaten niederschlagen, führten wir eine großflächige vierwöchige Studie der Prognostizierbarkeit durch. Ergebnis war, dass erstaunlich wenige Ampeln in einem für die Prognose problematischen Rahmen adaptiv schalten. Diese zentrale Erkenntnis stellt nicht nur einige vergangene Arbeiten in Frage, die direkt von einer hohen Anzahl verkehrsabhängig schaltbarer Ampeln auf eine herausfordernde Situation für die Prognose schlossen. Sie erklärt auch, warum die gemessene Prognosequalität höher war als erwartet. An dieser Stelle könnten zukünftige Arbeiten nahtlos anknüpfen, und ähnliche Data-Minings durchführen, um die Schaltverhalten von Ampeln noch besser zu verstehen. Dass hieran weiter geforscht werden sollte ist aus Sicht des potenziellen Erkenntnisgewinns für zukünftige Prognoseverfahren in dieser Arbeit deutlich geworden. Die Arbeit an temporal stabilen sowie gegenüber Datenlatenzen und Datenausfällen robusten Prognoseverfahren ist angesichts der Erkenntnisse dieser Arbeit ein weiterer zentraler Aspekt, der bisher kaum Aufmerksamkeit erhalten hat.

Das Ampelmatching war die zweite Hauptherausforderung, die gelöst werden musste, um eine GLOSA App für Radfahrer umsetzen zu können. Es war relativ schnell klar, dass sich diese Komponente am besten über die Georeferenzierung der Abbiegespuren der Ampeln auf MAP-Topologien umsetzen lässt. Kamera-basierte Methoden, wie sie in verwandten Arbeiten vor allem im autonomen Fahren verwendet werden, wurden relativ frühzeitig ausgeschlossen, da es derzeit keine praktikable Möglichkeit für deren Umsetzung bei Radfahrern gibt. Positionsbasierte Methoden zur Lokalisation des Fahrzeugs auf der passenden Spur scheinen bisher die beliebteste Variante in der Forschung rund um GLOSA zu sein, bringen aber eine Reihe von bekannten Nachteilen mit sich. Zum einen gehen diese Methoden von einer ausreichend langen Ingress-Spurgeometrie aus. Die Einfahrtsspurgeometrien für Radfahrer in Hamburg, das konnte durch diese Arbeit gezeigt werden, waren aber nicht annähernd so lang, wie man sie für eine sinnvolle Umsetzung dieses Verfahrens benötigt hätte. Als mögliche Erklärung wurde die Zufahrtsrichtung auf eine Ampel beschrieben, die bei Fahrradfahrern manchmal mehrdeutig sein kann. Es bräuchte also an vielen Kreuzungen mehrere Einfahrtsspurgeometrien, damit dieser Ansatz für Radfahrer umsetzbar wäre. Andererseits hatte sich in verwandten Arbeiten bereits die GNSS-Genauigkeit als problematisch herausgestellt, da sie in mehreren Studien zu einem Mismatching auf die falsche Spur führte.

Aus diesem Grund wurde eine dritte, bisher in verwandten Arbeiten nur als grobe Idee aufgegriffene, Variante weiter exploriert: ein routenbasiertes Ampelmatchingverfahren, welches die Spurgeometrien mit der Routengeometrie vergleicht, um passende Ampeln auszuwählen. Durch Anwendung spezieller geometrischer Vergleichsfilter und eines Machine Learning Modells konnte ein relativ genaues Matching entwickelt werden, trotz dessen, dass die verwendeten OpenStreetMap Routen manchmal deutlich von der Fahrradspur abwichen und somit keine einfache Entscheidung zuließen. Durch die Vorselektion der Ampeln ergeben sich noch deutlich mehr Vorteile, beispielsweise zur Maximierung der Aktivierungsdistanz oder zur späteren Auswertung der Trajektorien, bei denen die Positionen der ausgewählten Ampeln wiederverwendet wurden. Die Portierung auf andere Städte sollte aus jetziger Sicht ebenfalls kein Problem darstellen, da sowohl MAP-Topologien als auch OpenStreetMap in den meisten Städten verfügbar sind. 

Natürlich gibt es aber auch hier noch Potenzial für Verbesserungen. Wie sich später in der Nutzerstudie herausstellen sollte, gibt es vereinzelte Situationen, in denen das Ampelmatching eine falsche Ampel auswählt. Diese Situationen wurden von Nutzern nicht als problematisch beurteilt, aber dennoch könnten diese noch weiter reduziert werden. Generell 


