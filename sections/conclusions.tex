\chapter{Conclusions and Future Work}\label{ch:conclusions}

In this work, we addressed four specific challenges in implementing a GLOSA system for cyclists: traffic light prediction, traffic light matching, accurate bike routing, and speed advisory. For these challenges, the following key conclusions can be derived.

\textbf{\color{cidarkblue}1. Centralized traffic light data platforms are attractive for smartphone applications but may require rigorous data quality assurance and monitoring.} 

Looking at related works on traffic light data and prediction, we encountered a heterogeneous landscape of currently available systems and concepts under research. Among other options, centralized data infrastructures remain an attractive option for smartphone applications of GLOSA. Such a centralized data infrastructure was also utilized in this work: the Traffic Lights Data platform of Hamburg. By recording the traffic light states of multiple thousand traffic lights in parallel over MQTT streams, we established a data foundation for our prediction algorithm. However, our long-term study revealed significant challenges with data outages and errors. Investigations into recurring error patterns and a developed monitoring system were used to identify and resolve systematic issues in the data infrastructure. Particularly, individual traffic controllers, through which data was extracted, seemed to repeatedly cause failures in the data chain. These issues could be ultimately resolved, but only after long-term active collaboration with the platform operators.

\textbf{\color{cidarkblue}2. A probabilistic prediction method is attractive, as it is scalable, robust against data errors, and conveys uncertainties in the prediction.} 

Achieving a fully reliable and always available prediction is technically challenging, not only due to data errors. Additionally, adaptive traffic lights can affect the prediction with their more spontaneous switching behavior. Instead of a self-adaptive prediction approach, we reused a probabilistic prediction method from another work, which can differentiate between safe and unsafe areas of the prediction in the case of traffic-dependent switching. This method has already been shown in related works to be a good option for unstable and sometimes heavily delayed data and accurately represents the prediction's uncertainty areas without significant temporal fluctuation. Integrating this method demonstrated that it achieved good prediction quality at most traffic lights, while the prediction availability was often not optimal due to the aforementioned data issues.

\textbf{\color{cidarkblue}3. Traffic lights may be less affected by traffic adaptivity than previously assumed.} 

We were surprised by the accuracy of our prediction algorithm, as past works announced that a high number of traffic-adaptive signals also implies poor predictability. Especially in Hamburg, where most traffic lights have at least the capability to switch adaptively, we would have expected a worse performance. To further pursue this understanding and better comprehend which traffic-dependent switching patterns are actually reflected in the traffic light data, we conducted an extensive four-week study of predictability. The result was that remarkably few traffic lights switch adaptively to a problematic extent for the prediction. This central finding not only questions some past works, which directly inferred a challenging situation for the prediction from a high number of traffic-adaptive traffic lights, but also explains why the measured prediction quality was higher than expected. Future works could build on these findings and conduct similar data mining to better understand the switching behavior of traffic lights. The necessity for further research in this area has become evident. Working on temporally stable and robust prediction methods against data latencies and failures is another central aspect that has received little attention so far, given the insights of this work.

\textbf{\color{cidarkblue}4. Cross-checking traffic light predictions with user trajectories may be one option for future work to further identify faulty traffic lights.} 

Due to the repeated issues with the traffic light data, we were highly sensitized to potential errors in the traffic light data. One issue for which we do not have a proper way of measuring is the accuracy of the real-time traffic light data itself. All conducted prediction accuracy evaluations are based on the assumption that the contained timestamps in the traffic light data are correct. However, since we know that there were previous issues with latencies in related work or also in the Traffic Lights Data platform, even if the timestamp for each real-time message is generated directly in the intersection controller, minor internal processing delays or unsynchronized clocks may invalidate this information. To further investigate on a large scale how real this potential issue may be, data from the user trajectories may be utilized to infer which traffic lights potentially send faulty data. Such an approach was identified as a Diploma thesis topic and investigated by Florian Pix \cite{pix_2024}, showing promising initial results.

\textbf{\color{cidarkblue}5. To automatically identify traffic lights along the cyclist's trajectory, existing methods come with significant limitations.} 

Traffic light matching was the second main challenge that needed to be solved to implement a GLOSA app for cyclists. It quickly became apparent that this component could best be implemented through georeferencing the turning lanes of the traffic lights on MAP topologies. Camera-based methods, predominantly used in related works for autonomous driving, were ruled out relatively early on, as there is currently no feasible way to implement them for cyclists. Position-based methods for locating the vehicle on the appropriate lane seem to be the most popular variant in the research around GLOSA, but they come with a number of known drawbacks. First, these methods assume a sufficiently long ingress lane geometry. However, as shown by this work, the ingress lane geometries for cyclists in Hamburg were not nearly as long as would be needed for a meaningful implementation of this approach. One possible explanation described was the approach direction to a traffic light, which can sometimes be ambiguous for cyclists. Thus, there would need to be multiple ingress lane geometries at many intersections for this approach to be feasible for cyclists. On the other hand, GNSS accuracy had already been identified as problematic in related works, as it led to mismatching onto the wrong lane in several studies.

\textbf{\color{cidarkblue}6. A route-based traffic light matching is possible, even in the presence of an inaccurate depiction of bike infrastructure, by comparing the route with traffic light turn geometries.}

Due to the issues with location-based approaches, a third, previously only roughly discussed option in related works was further explored, which compares lane geometries with the route's geometry to select suitable traffic lights. By applying special geometric comparison filters and a machine learning model, a relatively accurate matching could be developed, despite the fact that the used OpenStreetMap routes sometimes deviated significantly from the bike lane and thus did not allow for a straightforward decision. Preselecting the traffic lights offers even more advantages, for example, to maximize the activation distance or for later evaluation of trajectories, where the positions of the selected traffic lights were reused. From the current perspective, porting to other cities should also not be a problem, as both MAP topologies and OpenStreetMap are available in most cities. However, there is still potential for improvement here as well. As later revealed in the user study, there are isolated situations where the traffic light matching selects the wrong traffic light. Users did not perceive these situations as problematic, but they could still be further reduced. Generally, it can be said that, according to the results of this work, the route-based approach offers promise, possibly also for automotive applications.

\textbf{\color{cidarkblue}7. Combining authoritative bike routing datasets with OpenStreetMap may be an option to address its issues with quality and consistency.}

Since the developed GLOSA approach heavily relies on the quality and accuracy of the route through route-based matching, the third challenge was to explore ways to further improve bike routing. After analyses from related works, other partially commercial routing providers like Google Maps performed similarly to OpenStreetMap. For this reason, a third option was explored: bike routing based on an authoritative dataset. The DRN in Hamburg represented such a dataset. Subsequently, a concept was developed primarily focusing on the fully automated processing of this dataset and integration into a routing service. With a focus on reusability with other authoritative routing datasets, the DRN model was initially converted into the OpenStreetMap format. However, the fact that bike paths on both sides of the road were now present instead of a single geometry in the middle of the road caused unnecessary detours. The solution involved allowing routing on each contained path in the opposite direction with walking speed. A concept for stitching at the dataset boundary was also implemented to enable routing across the city border. Reuse of the concept would involve adapting the metadata and geometry transformer to the respective format. In several iterations, different cities or regions could be exchanged in OpenStreetMap with the authoritative data.

\textbf{\color{cidarkblue}8. Employed in a GLOSA application, bike routing on authoritative datasets may benefit not only routing accuracy and error freeness but also traffic light matching and distance-to-signal estimation.} 

The evaluation of the developed bike routing showed a substantial improvement over OpenStreetMap, but also over Google Maps and Bing Maps. In alignment with actual cycling infrastructure, the DRN routing performed better, as well as in the convergence of routes with a specific route request. In addition to OpenStreetMap, Google Maps and Bing Maps had the problem, even more pronounced, that they mostly route on the road instead of on the bike path. This aspect is especially problematic for the selection of bike traffic lights. Through improved alignment with the cycling infrastructure, the traffic light matching on DRN was also improved by several percentage points, while it also better matches the GNSS trajectories of the users. This aspect is relevant for snapping to correct the app's GNSS position and estimating the distance to the traffic light. With an accurate measurement of distance estimation, it could be shown that a route-based distance estimation remains more stable with increasing distance to the traffic light, while a straight-line-based estimation tends to underestimate the distance further away due to the lack of road curves. Such an estimation was used in some works, while methodological details on distance estimation often remain open, although it is an essential part of all GLOSA systems that do not only display a countdown.

\textbf{\color{cidarkblue}9. Further studies may be needed to generalize our results to other authoritative routing datasets.} 

The positive results regarding bike routing should motivate future work to implement a similar approach for other cities or regions. After all, the reusability of bike routing is not limited to GLOSA applications. However, it should also be noted that with the DRN dataset, there was already a complete routable graph available, including normal roads with no bike paths. Developments of the approach used in this work would be necessary where completeness is not given.

\textbf{\color{cidarkblue}10. Projective methods for the route-enhanced speed advisory are a good option to circumvent the issue of insufficient context awareness with target speed recommendations.} 

In the final chapter, we dedicated ourselves to the overall integration of the approaches into the app and focused on the user interface of the speed recommendation. In previous works, a wide variety of user interfaces were found, primarily relying on the visual communication of the speed recommendation. The recommendation of a target speed, countdowns, and projective methods were most commonly used. Especially with the recommendation of a target speed, it turned out in a previous bike study that it is often not context-sensitive enough. After testing various prototypes, this work chose the projective speedometer visualization in combination with a countdown. User feedback supports our assumption that projecting all traffic light states onto the speedometer allows users to feel free to select the speed. Overall, the inclusion of routing as a guidance in the ride view was also positively evaluated. In addition, users felt not very distracted or encouraged in riskier driving behavior. Here, of course, it needs to be considered that the users' self-evaluation may have been biased. Thus, one option in the future would be to conduct more objective distraction studies using a front-facing (smartphone) camera to see how long cyclists focus on the smartphone screen, similar to the study by Krause et al. (2012) \cite{krause_traffic_2012}. After this idea was discovered, the technical foundation for such an experiment was laid by Markus Wieland \cite{wieland_2023} in his Diploma thesis.

\textbf{\color{cidarkblue}11. Alternative interaction methods should be explored, such as auditive speed advisories.} 

In our user study, although the speed advisory user interface was generally received well, we could clearly identify a potential for alternative speed advisory user interfaces. Some users simply wanted to stow away their smartphones and listen to the speed advisory instead of being required to install and monitor the smartphone on an additional handlebar mount. Due to these reasons and the potential to further decrease distraction from traffic, an auditive speed advisory user interface presents an attractive pathway for further research. Studies that incorporate auditive signals in their speed advisory concept can be utilized as an entry point for future work \cite{suramardhana_driver-centric_2014, xu_bb_2015, wilson_driver_2017, sokolov_effects_2018, zhang_green_2020, chen_developing_2022}. On this avenue, other challenges must be solved, such as finding a suitable sound or text-to-speech concept that strikes a balance between too few or too many advisories, resulting in annoyance. Here, novel research results such as by Gedicke et al. (2023) \cite{gedicke_selecting_2023} could be incorporated to combine the speed advisory with reduced but meaningful auditive navigation instructions. Combining the navigation instructions with the speed advisory, as demonstrated with our visual user interface, may also address the challenge of explaining to the user for which traffic light the speed advisory is presented. Such an approach was identified as a Diploma thesis topic and investigated by Paul Pickhardt \cite{pickhardt_2023}.

\textbf{\color{cidarkblue}12. Bike-GLOSA measurably improves rides, but the effect is twofold.} 

A study of the results of related works revealed a wide range of possible stop reductions or energy savings with the speed recommendation. However, larger-scale studies conducted in real cities dampen the expectations from studies on test tracks or simulations. So far, there have been hardly any concrete numbers on the impact on cyclists. Against this background, it is unsurprising that the larger-scale study in this work significantly dampens previous assessments. The most significant effect can be seen in the reduced stop rate if a recommendation is followed, with the recommendation being reachable in this case. A key insight from the analysis is the distinction between cases with adherence and without adherence to the speed advisory. The results should be viewed in the context of the not-quite-ideal data availability and prediction reliability during the test period. Possibly even more pronounced effects could be achieved if these aspects could be improved. The long-term societal effects of the developed system, or other future bike-GLOSA applications, also remain an open field for research. The key question with which future work should deal is whether bike-GLOSA apps lead to a long-term increase in cycling or can sustainably motivate people to cycle. To draw such conclusions on the long-term motivation, the study scale and time frame would likely have to be significantly increased. This is one of the learnings from Mattis Lahr's Master thesis \cite{lahr_2023}, which was also identified and supervised during this study.

\begin{Summary}[Outlook]
This work demonstrated how a bike-GLOSA system can be implemented in a city like Hamburg. The results are already promising, with good app usability and measurable improvements in cycling. However, further improvement opportunities can also be identified, mainly in data availability and prediction reliability. The logical next step, scaling to multiple cities, would also contribute to the ultimate goal of permanently motivating more people to cycle. Furthermore, GLOSA apps should not only be thought unidirectionally from traffic light to cyclist but also in the other direction. Allowing users to influence the traffic light's switching behavior while riding toward the traffic light could further benefit an enhanced ride experience.
\end{Summary}